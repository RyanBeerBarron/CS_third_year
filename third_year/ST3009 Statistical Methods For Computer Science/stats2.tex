\documentclass{article}
\title{ST3009 Weekly Questions 2}
\author{Ryan Barron // Student number: 16329561}
\date{}
\usepackage{amsmath}
\usepackage{mathtools}
\begin{document}
\maketitle

\paragraph{Question 1}
\subparagraph{a)}   
Each die can roll 6 different number and we roll 3 die, so the total number of outcomes is $6^3$ which is equal to 216.
\subparagraph{b)}
The set of outcomes with at least one 2 rolled is the same as the total set of outcomes minus the all the outcomes with no 2 in them. In terms of probability, it means that the probability of at least one 2, is 1 minus the probability of no 2 rolled: $1 - \frac{5^3}{6^3}=0.421$
\subparagraph{c)}
The \textsc{Matlab} simulation gives a similar result
\subparagraph{d)}
There are 3 possibilities for the dice rolls to sum up to 17, there is (6,6,5), (6,5,6) and (5,6,6). So the probability for that is $\frac{3}{216} = 0.013888$
\subparagraph{e)}
If the first roll was a 1, it means that the two other rolls need to add up to 11, and so that gives us (5,6) and (6,5) but this time, out of only $6^2$ possibilities, so the probability for that is $\frac{2}{36} = 0.0555$
\paragraph{Question 2}
\subparagraph{a)}
The probability for the second roll to be a five is sum of the probability of first rolling a 1 and then a 5 using the 6 sided die, and the probability of rolling anything but a 1 and finally rolling a 5 using the 20 sided die. That is because these two cases are mutually exclusive because of the first roll. So we get:
\begin{equation*}
\begin{split}
P(Second\: throw\: is\: 5) & = \frac{1}{6}\times\frac{1}{6}+\frac{5}{6}\times\frac{1}{20} \\ 
P(Second\: throw\: is\: 5) & = 0.0694	
\end{split}
\end{equation*}
\subparagraph{b)}
Same process here but since we want the second roll to be equal to 15, only the 20 sided die can achieve such a roll, so the probability to roll 15 if the first roll is 1, is equal to 0 and we can just ignore that probabibility, so we get
\begin{equation*}
\begin{split}
P(Second\: throw\: is\: 15) & = 0 +\frac{5}{6}\times\frac{1}{20} \\ 
P(Second\: throw\: is\: 15) & = 0.0416	
\end{split}
\end{equation*}
\paragraph{Question 3}
\subparagraph{}
We want to know the probability of the suspect being guilty, knowing he has the characteristic, we shall write as P(G|C). We'll use Bayes rule for that, and so we need to know the probability of P(G), that is 60\%, the probability of him having the characteristic, knowing he is guilty. Well for that we know the actual criminal has it, so if he is guilty, he has a 100\% of having the characteristic. Finally we need to calculate the probability P(C), for that we'll use marginalisation.
So we get:
\begin{equation*}
\begin{split}
P(G|C) & = \frac{P(C|G)P(G)}{P(C)}, s.t P(C) = P(C|G)P(G)+P(C|\lnot G)P(\lnot G) \\ 
P(G|C) & = \frac{1*0.6}{1*0.6 + 0.2*0.4} = 0.88
\end{split}
\end{equation*}
\paragraph{Question 4}
\subparagraph{}
\[
\begin{bmatrix}
0.0744 & 0.1885 & 0.0744 & 0.005 \\ 0.005 & 0.1488 & 0.0942 & 0.0744 \\ 0.001 & 0.005 & 0.1488 & 0.0942 \\ 0.001 & 0.001 & 0.0099 & 0.0744
\end{bmatrix}
\]
I get this grid matrix for the probability of having two bar signal, knowing the location. To get this, I used Bayes rule for each of the grid cell. If L\textsubscript{i, j} corresponds to the j\textsuperscript{th} cell of the 
i\textsuperscript{th} row for the location grid and S\textsubscript{i, j}\textbar L\textsubscript{i, j} correspond to the i, j cell on the other grid, we get the following equation:
\begin{equation*}
\begin{split}
P(L\textsubscript{i, j}|S\textsubscript{i, j}) = \frac{P(S\textsubscript{i, j}|L\textsubscript{i, j})P(L\textsubscript{i, j})}
{P(S)}, \forall i, j\;s.t\;1\leq i, j\leq 4
\end{split}
\end{equation*}
Where P(S) is the probability of having signal, at all. This is calculated by using marginilisation and summing up
all the probabilitiies of having two bar of signal in a given location, for all the location. All the probabilities from the left grid a mutually exclusive, someone can not be in two locations at the same time. So for P(S), we get:
\begin{equation*}
\begin{split}
P(S) = \sum_{i, j=1}^{4} P(S\textsubscript{i, j}|L\textsubscript{i, j})\times P(L\textsubscript{i, j}) = 0.504
\end{split}
\end{equation*}	
\end{document} 